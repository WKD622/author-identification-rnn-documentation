\newpage
\section{Organizacja pracy}

\subsection{Metodyka}

Prace nad projektem były prowadzone zgodnie z iteracyjnym modelem wytwarzania oprogramowania. 
Większość iteracji opierała się na następującym schemacie: 
\begin{enumerate}
  \item wybranie funkcjonalności do zaimplementowania,
  \item zdobycie informacji na temat tej funkcjonalności,
  \item implementacja funkcjonalności,
  \item testowanie funkcjonalności,
  \item stworzenie dokumentacji na temat tej funkcjonalności.
\end{enumerate}
Każda iteracja, ze względu na odmienny charakter trwała inną ilość czasu. Najdłuższa okazała się
trzecia i ostatnia iteracja. W skład trzeciej wchodziło zapoznanie się z tematyką sieci typu RNN oraz
jej implemetacja, w skład ostatniej uczenie sieci oraz szukanie najlepszych parametrów sieci do problemu 
identyfikacji autorstwa tekstu.

\subsection{Role i podział prac}
Bazując na zadaniach wykonywanych podczas pracy nad biblioteką, można wyróżnić następujące role:
\begin{itemize}
  \item Programista - implementacja kodu
  \item Analityk -  bezpośredni kontakt z klientem i określanie wymagań odnośnie tworzonej biblioteki
  \item Tester - testowanie kodu napisanego przez programiste
  \item Osoba odpowiedzialna za tworzenie dokumentacji
\end{itemize}
Projekt wykonywany był przez dwie osoby:
\begin{itemize}
  \item Jakub Ziarko - implementacja modułu preprocessingu, implementacja modułu sieci RNN, 
  trenowanie sieci, pisanie dokumentacji, stworzenie skrytpów ułatwiających prace na Prometheusie.
  \item Jan Liberacki - implementacja modułu odpowiedzialnego za podział danych na batche, 
  implementacja modułu sieci RNN, trenowanie sieci, pisanie dokumentacji, testowanie kodu.
\end{itemize}
Opiekunowie pracy:

\begin{itemize}
  \item dr inż. Marcin Kuta - główny opiekun pracy, pełniący rolę klienta końcowego i osoby nadzorującej
  postęp prac nad realizacją i implementacją projektu. Z tą osobą zespół konsultował wszelkie wątpliwości
  natury technicznej. Dzięki jego uwagom doprowadzono do wielu usprawnień w zakresie działania sieci.
  \item dr inż. Rafał Dreżewski - prowadzący Pracownię Projektową, nadzorował postęp prac, 
  szczególnie ten w dokumentacji. Dzięki jego uwagom udało się dokonać wielu usprawnień w zakresie podziału
  dokumentacji oraz jej formy.
\end{itemize}

\subsection{Komunikacja z opiekunem pracy}
W trakcie pracy nad biblioteką odbywały się regularne spotkania Pracowni Projektowej, w ramach której 
przedstawiane były kolejne postępy pracy nad biblioteką. W ramach tego przedmioru byliśmy również
zobligowani do dostarczania kolejnych części dokumentacji na określone terminy. Tempo pisania kolejnych modułów
samo wyznaczało kolejne spotkania z głównym opiekunem pracy. Często w ramach przejścia do kolejnej iteracji
pojawiały się nowe problemy, które wymagały pomocy lub dyskusji. Większość pytań kierowanych do 
głównego opiekuna była realizowana drogą mailową. 

\subsection{Narzędzia}
Sprawna realizacja projektu wymagała użycia odpowiednich narzędzi. Wśród wielu z nich
chcielibyśmy wyróżnić te najważniejsze:

\begin{itemize}
  \item PyCharm - zintegrowane środowisko programistyczne dla języka Python od firmy JetBrains.
  Wybrane przez nas ze względu na nasze wcześniejsze pozytywne doświadczenia.
  \item Git - rozporoszony system kontroli wersji. Zdecydowaliśmy się na niego ponieważ ze względu na 
  dwuosobowy zespół zależało nam na możliwości współbieżnej pracy nad kodem i dokumentacją.
  \item GitHub - służył jako repozytorium do przechowywania dokumentacji oraz kodu.
  \item Facebook - główne narzędzie komunikacji pomiędzy członkami zespołu.
  \item Latex - oprogramowanie do zautomatyzowanego składu tekstu, a także związany z nim język 
  znaczników, służący do formatowania dokumentów tekstowych i tekstowo-graficznych. 
  \item Eclipse - zintegrowane środowisko programistyczne do tworzenia programów, wybrane przez nas 
  w celu tworzenia dokumentacji.
  \item TeXlipse - rozbudowana wtyczka do środowiska Eclipse ułatwiająca pisanie w Latex.  
  \item Matplotlib - biblioteka Pythonowa do rysowania wykresów, wykorzystana przez nas do analizy danych 
  z procesu uczenia sieci. 
  \item Draw.io - aplikacja webowa do rysowania diagramów, korzystaliśmy z niej do tworzenia 
  schematów do dokumentacji.
\end{itemize}

\subsection{Zastosowane techniki i praktyki}
Realizacji biblioteki towarzyszyło zastosowanie kilku technik i praktyk mających na celu efektywne 
tworzenie kodu o dobrej jakości:
\begin{itemize}
  \item refactoring kodu - nie wszystkie moduły biblioteki od początku napisane były dobrze. Niekiedy z racji 
  na kolejne optymalizacje bądź korekcje w sposobie uczenia sieci należało szybko coś zmienić w kodzie,
  co z czasem owocowało brakiem porządku i kodem niskiej jakości. Okresowe refactoringi pozwoliły temu zapobiec.
  \item zlecenie dołączenia oraz analiza kodu - ze względu na złożoność modułów i potrzebe utrzymywania wysokiej
  jakości kodu zmiany wprowadzane w repozytorium były zazwyczaj poprzez tworzenie zleceń dołączenia oraz analizy kodu
  drugiego członka zespołu. W ten sposób obaj na bieżąco zapoznawaliśmy się ze zmianami oraz zwiększaliśmy
  jakość kodu, poprzez efektywniejsze wychwytywanie błędów oraz złych wzorców projektowych.
\end{itemize}

\subsection{Postępy}
W ramach tworzenia pracy inżynierskiej wyodrębniliśmy następujące iteracje:
\begin{enumerate}
  \item Zaimplementowanie modułu preprocessingu.
  \item Zaimplementowanie modułu batchowania.
  \item Zaimplementowanie modułu sieci.
  \item Zaimplementowanie modułu treningu sieci.
  \item Stworzenie skryptów ułatwiających szkolenie sieci na Prometheusie oraz analizę wyników.
  \item Zbieranie wyników, ich analiza oraz poprawianie parametrów sieci.
\end{enumerate}

\subsection{Wykonane prace}
W trakcie realizacji pracy inżynierskiej wykonaliśmy następujące prace:
\begin{itemize}
  \item stworzenie modułu preprocessingu - moduł który w wygodny sposób pozwala nam na przygotowanie tekstów 
  do uczenia maszynowego. 
  \item stworzenie modułu batchowania - moduł, który bazując na danych przygotowanych przez moduł 
  preprocesingu dzieli je na mini batche i dostarcza sieci. Jest on sparametryzowany dzięki czemu mogliśmy
  w łatwy sposób wpływać na parametry batchy i efektywnie szukać tych najlepszych dla wyników sieci. 
  \item stworzenie modułu sieci - moduł, który tworzy rekurencyjną sieć neuronową za pomocą
  biblioteki PyTorch.
  \item stworzenie modułu treningu - moduł, który pozwala na wytrenowanie rekurencyjnej sieci neuronowej. 
  Współpracuje bezpośrednio z modułem batchowania.
  \item stworzenie modułu do przetwarzania wyjścia z sieci - odpowiedzialny za zbieranie informacji podczas procesu uczenia,
  następnie wybierania część z nich i umieszcza je w uporządkowanej formie w pliku csv oraz 
  jako standardowe wyjście. Jest on również odpowiedzialny za zapisywanie kolejnych wytrenowanych modeli. 
  \item stworzenie skryptów do pracy na Prometheusie:
  \begin{itemize}
    \item skrypt do wysyłania przetworzonych tekstów na Prometheusa 
    \item skrypt do kopiowania danych po procesie uczenia maszynowego z Prometheusa na komputer osobisty
    \item skrypt do rysowania wykresów na podstawie danych zerbanych w procesie uczenia sieci
    \item skrypt do definiownia parametrów sieci, których szkolenie było następnie w sposób z
    automatyzowany przeprowadzane na Prometheusie. Dzięki niemu nie musieliśmy pilnować procesu uczenia i 
   	każde z nich przeprowadzać osobno. Definiowaliśmy parametry wielu sieci, które chcieliśmy przetestować,
   	następnie skrypt generował kod który kolejkował na Prometheusie uczenie wszystkich wyszczególnionych przez nas sieci 
   	jako kolejne zadania.
  \end{itemize}
  \item wykonanie niniejszej dokumentacji
  \item zrozumienie działania rekurencyjnych sieci neuronowych oraz dobranie właściwych parametrów sieci 
  w celu uzyskania lepszych wyników w zakresie indentyfikacji autorstwa tekstu w języku angielskim.
\end{itemize}


\subsection{Problemy napotkane w trakcie realizacji projektu}
Podczas tworzenia biblioteki napotkaliśmy na wiele problemów. Zdecydowaliśmy się je opisać z podziałem na 
najbardziej problematyczne iteracje:

\begin{enumerate}
  \item Dużym wyzwaniem podczas pierwszej iteracji okazało
  się zaprojektowanie modułu preprocessingu. Moduł ten okazał się być bardzo obszerny w porównaniu do
  pozostałych części biblioteki. W związku z tym należało przedsiewziąć odpowiednie decyzje projektowe 
  już na samym początku, a kod dokładnie zaplanować.
  
  \item Druga iteracja, podczas której tworzony był moduł preprocessingu okazała się trwać przez praktycznie
  cały okres pisania biblioteki. Ten moduł bowiem, biorąc za wejście przetworzone teksty w postaci tensorów,
  przetwarza ich ``formę'' na taką, która nadaje się do efektywnego szkolenia sieci. Podczas pisania pracy 
  nasze koncepcje na temat tego procesu wielokrotnie się zmieniały w związku z tym zmieniał się również
  kod modułu batchowania. Początkowe, błędne decyzje projektowe sprawiły, że wielkorotnie musiał być 
  napisany od początku. Te doświadczenia były dla nas bardzo cenne i nauczyły nas, żeby poświęcać
  więcej czasu na planowanie kodu przed właściwym jego pisaniem i stosować się do wzorców projektowych.

  \item Trzecia iteracja, a więc zaimplementowanie modułu sieci okazało się wysoce problematyczne ze względu na obszerność 
  wiedzy potrzebnej do tego aby poruszać się w tematyce rekurencyjnych sieci neuronowych, ale także
  na późniejszy problem przełożenia tej wiedzy na kod w bibliotece PyTorch.
  
  \item Piąta iteracja wymagała od nas dużej liczby eksperymentów w postaci szkolenia sieci z wykorzystaniem wybranych
  parametrów, a następnie na podstawie otrzymanych rezultatów ponowne dobieranie parametrów aby rezultaty poprawić.
  Najbardziej problematyczny okazał się czas uczenia, pomimo wykorzystania grantu wyszkolenie sieci trwało
  wiele godzin.
\end{enumerate}
