\subsection{Funkcja kosztu}
Funkcja kosztu mierzy wydajność modelu dla podanych danych wejściowych.
Funkcja kosztu określa błąd pomiędzy predykowanymi a spodziewanymi wartościami i wyraża go jako pojedynczą 
liczbę rzeczywistą. 
Określamy ją jako:

\begin{align*}
  &L = - \frac{1}{S_y}\sum_{i=1}^{S_y}[t_i\ln(a_i) + (1-t_i)\ln(1-a_i)]
\end{align*}

gdzie: \newline
$L$ - funkcja kosztu (krosentropia), \newline
$t$ - wektor wartości oczekiwanych, reprezentuje on kolejny znak w sekwencji, \newline
$a$ - wektor wartości otrzymanych, reprezentuje prawdopodobieństwo kolejnego znaku,  \newline
$S_y$ - wartość wektora wyjściowego, \newline
