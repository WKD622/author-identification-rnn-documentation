%##noBuild
\newpage
\section{Cel i wizja produktu}

\subsection{Definicja problemu}
Zadanie weryfikacji autorstwa tekstu to ocena czy autor danego zbioru tekstów jest również 
autorem kwestionowanego tekstu. Weryfikacja autorstwa ma się odbywać dla języków: angielski, 
hiszpański, holenderski, grecki. Każdy z tekstów zawiera od kilkuset do kilku tysięcy liter.
 
\subsection{Cel pracy}
Celem projektu jest stworzenie biblioteki zdolnej do tworzenia i trenowania sieci dla zadania 
weryfikacji autorstwa tekstu pisanego. Praca ta ma rozwijać pracę inżynierską z poprzednich lat, 
która nie wyczerpała tego tematu. Osiągnięto pozytywne wyniki dla języka angielskiego, natomiast 
dla pozostałych z wyżej wymienionych języków są one nadal niezadowalające.


\subsection{Motywacja do pracy}
Motywacja do projektu nie zmieniła się znacznie względem pracy na której bazujemy. Mimo pojawienia 
się wielu zaawansowanych narzędzia do łatwego tworzenia nawet bardzo skomplikowanych sieci, 
zapewniających przy tym szybką naukę oraz kontrolę nad jej przebiegiem wciąż brakuje dopracowanej 
biblioteki, która pozwoliłaby na tworzenie i trenowanie sieci dla zadania weryfikacji autorstwa 
tekstu. Projekt, którego dalszego rozwoju się podejmujemy nie w wyczerpał tematu, stąd pojawia się 
miejsce dla tej pracy. Największa różnicą względem pracy, na której już bazujemy jest fakt, że 
naszym celem jest zwiększenie dokładności tego rozwiązania.

\subsection{Wizja}
Wizją tej pracy jest stworzenie dopracowanej biblioteki pozwalającej na konfigurowalne tworzenie 
głębokich sieci typu RNN, trening, preprocessing danych wejściowych. Oprócz tego powinna ona 
zapewnić persystencję wytrenowanych modeli oraz ich dalsze testowanie. System powinien wspierać 
naukę w czterech językach: angielskim, greckim, hiszpańskim oraz holenderskim. Grecki, hiszpański 
oraz holenderski wymagają największego dopracowania. Wyżej wspomniana konfigurowalność powinna 
pozwalać na tworzenie dowolnych sieci parametryzowanych hiperparametrami: głębokość sieci, wymiar 
wektorów wejściowych, liczba neuronów w warstwach ukrytych, wielkość mini-batchów, itd. Ważna jest 
też możliwość wizualizacji modelu, np. poprzez generację grafu obrazującego strukturę sieci. 
Biblioteka powstanie w języku Python przy wykorzystaniu biblioteki PyTorch.

\subsection{Analiza ryzyka}

\subsubsection{Czas ewaluacji i eksperymentów}
Stworzenie podobnej biblioteki wiąże się z nieustannym treningiem oraz testowaniem jakości 
generowanych sieci, stąd jedną z przeciwności będzie czas, który będzie potrzebny na wszystkie 
obliczenia. Będą one wymagać dużej ilości czasu procesora. W tym wypadku wszystkie obliczenia 
będą prowadzone z wykorzystaniem Prometeusza.


\subsubsection{Wiedza na temat sieci neuronowych}
Sieci neuronowe są bardzo złożoną kwestią, która wymaga sprawnego poruszaniem się w wyższej 
matematyce. W szczególności w kontekście tego projektu potrzebne będzie bardzo dokładne zgłębienie 
tego tematu. Jest to dla nas największa przeszkoda. Wymaga czasu oraz dużej cierpliwości, 
szczególnie że jest to dla nas zupełnie nowe zagadnienie.


\subsubsection{Znajomość języków programowania oraz bibliotek }
Obojga autorów tej pracy cechuje wysoka znajomość języka Python. Natomiast przeszkodą będzie 
opanowanie biblioteki PyTorch, która jest dla nas zupełną nowością.



\subsubsection{Wykonalność ekonomiczna }
Będziemy korzystać z darmowych bibliotek oraz środowiska, na które mamy licencje jako studenci. 
Ewentualne koszty będą wynikać z dostępu do wiedzy, np. kursy internetowe, książki.