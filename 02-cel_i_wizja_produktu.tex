%##noBuild
\newpage
\section{Cel i wizja produktu}

\subsection{Definicja problemu}
Zadanie identyfikacji autorstwa tekstu to ocena czy autor danego zbioru tekstów jest również 
autorem kwestionowanego tekstu. Inspiracją jest zadanie konkursu PAN 2015. Uczestnicy otrzymali skończony zbiór dokumentów pewnego autora
oraz jeden dokument, którego autorstwo nie było znane. Ich celem było odpowiedzenie na pytanie, czy dany autor napisał nieznany dokument.
Najskuteczniejszym rozwiązaniem okazało się użycie wielogłowicowych, rekurencyjnych sieci neuronowych Douglasa Bagnalla.

 
\subsection{Cel pracy}
Celem projektu jest stworzenie biblioteki do tworzenia i trenowania sieci dla zadania 
identyfikacji autorstwa tekstów. Praca ta ma rozwijać pracę inżynierską Marcina Radzio. Osiągnięto pozytywne wyniki dla języka angielskiego, natomiast 
dla pozostałych z wyżej wymienionych języków są one nadal niezadowalające.


\subsection{Motywacja do pracy}
Mimo pojawienia się wielu zaawansowanych narzędzi do łatwego tworzenia nawet bardzo skomplikowanych sieci, 
zapewniających przy tym szybką naukę oraz kontrolę nad jej przebiegiem wciąż brakuje biblioteki, dla zadania identyfikacji autorstwa 
tekstu.

\subsection{Wizja}
Wizją tej pracy jest stworzenie dopracowanej biblioteki pozwalającej na konfigurowalne tworzenie 
głębokich sieci typu RNN, trening, preprocessing danych wejściowych. Oprócz tego powinna ona 
zapewnić persystencję wytrenowanych modeli oraz ich dalsze testowanie. Biblioteka powinna wspierać 
naukę w czterech językach: angielskim, greckim, hiszpańskim oraz holenderskim. Wyżej wspomniana konfigurowalność powinna 
pozwalać na tworzenie dowolnych sieci parametryzowanych hiperparametrami: głębokość sieci, wymiar 
wektorów wejściowych, liczba neuronów w warstwach ukrytych, wielkość mini-batchów, itd. Ważna jest 
też możliwość wizualizacji modelu, np. poprzez generację grafu obrazującego strukturę sieci. 
Biblioteka powstanie w języku Python przy wykorzystaniu biblioteki PyTorch.

\subsection{Analiza ryzyka}

\subsubsection{Czas ewaluacji i eksperymentów}
Stworzenie biblioteki wiąże się z intensywnym treningiem oraz ewaluacją
generowanych sieci, stąd jednym z czynników ryzyka będzie mała ilość czasu, który będzie potrzebny na wszystkie 
obliczenia. Będą one wymagać dużej ilości czasu procesora. W tym wypadku wszystkie obliczenia 
będą prowadzone z wykorzystaniem Prometeusza.

\subsubsection{Wiedza na temat sieci neuronowych}
Sieci neuronowe są złożoną dziedziną nauki, która wymaga sprawnego posługiwania się formalizmem matematycznym.
Jest to dla nas największa przeszkoda. Opanowanie tej wiedzy wymaga czasu oraz dużej cierpliwości, 
szczególnie że jest to dla nas nowe zagadnienie.

\subsubsection{Znajomość języków programowania oraz bibliotek }
Przeszkodą może być opanowanie biblioteki PyTorch, która jest dla nas nowością.

\subsubsection{Aspekt ekonomiczny }
Będziemy korzystać z darmowych bibliotek oraz środowiska, na które mamy licencje jako studenci.