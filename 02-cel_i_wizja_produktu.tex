%##noBuild
\newpage
\section{Cel i wizja produktu}

\subsection{Definicja problemu}
W dzisiejszych czasach internet umożliwia dostęp do wielu publikacji autorstwa różnych osób.
Pozwala to na spojrzenie na problemy z wielu perspektyw, co pozytywnie wpływa na zapoznanie się z tematem.
Istnieje jednak niebezpieczeństwo, że czyjaś praca może zostać splagiatowana. Plagiat nie zawsze polega jednak
na prostym skopiowaniu cudzych słów. Zamiast porównywać teksty w celu weryfikacji autorstwa, 
spróbować można stwierdzić, czy model używanego języka zgodny jest z tym, którego normalnie używa osoba
podpisująca się pod pracą. Problem ten był tematem zadania konkursowego PAN 2015 \cite{pan}.
Najskuteczniejszym rozwiązaniem okazało się użycie wielogłowicowych, rekurencyjnych sieci 
neuronowych Douglasa Bagnalla \cite{bagnall}, dlatego zdecydowaliśmy się na jego użycie.

 
\subsection{Cel pracy}
Celem projektu jest stworzenie biblioteki do tworzenia i trenowania sieci dla zadania 
identyfikacji autorstwa tekstów. Praca ta jest rozwinięciem pracy inżynierskiej \cite{radzio}. Osiągnięto w 
niej pozytywne wyniki dla języka holenderskiego, natomiast dla pozostałych z wyżej wymienionych języków 
wciąż można poprawić rezultaty,co jest również celem tej pracy.


\subsection{Motywacja}
Mimo pojawienia się wielu zaawansowanych narzędzi do łatwego tworzenia nawet bardzo skomplikowanych sieci, 
zapewniających przy tym szybką naukę oraz kontrolę nad jej przebiegiem wciąż brakuje biblioteki dla zadania identyfikacji autorstwa 
tekstu.
Biblioteka weryfikująca autorstwo może posłużyć jako podstawa systemu antyplagiatowego. Systemy takie, 
jak na przykład Jednolity System Antyplagiatowy (JSA) używane są na wszystkich uczelniach w kraju.
Niedopuszczalne jest aby student otrzymał tytuł na podstawie pracy innej osoby. Innym zastosowaniem
wykonanej przez nas biblioteki może być idenftyfikacja autorstwa tekstów historycznych, którego autorstwo jest
nieznane na podstawie porównania ze stylami językowymi osób piszących w tamtym okresie czasu.

\subsection{Wizja}
Wizją tej pracy jest stworzenie dopracowanej biblioteki pozwalającej na konfigurowalne tworzenie 
głębokich sieci typu RNN, trening, oraz preprocesing danych wejściowych. Oprócz tego powinna ona 
zapewnić persystencję wytrenowanych modeli oraz ich dalsze testowanie. Biblioteka powinna wspierać 
naukę w czterech językach: angielskim, greckim, hiszpańskim oraz holenderskim. Wyżej wspomniana konfigurowalność powinna 
pozwalać na tworzenie dowolnych sieci parametryzowanych hiperparametrami: głębokość sieci, wymiar 
wektorów wejściowych, liczba neuronów w warstwach ukrytych, wielkość mini-batchów, itd. Ważna jest 
też możliwość wizualizacji modelu, np. poprzez generację grafu obrazującego strukturę sieci. 
Biblioteka powstanie w języku Python przy wykorzystaniu biblioteki PyTorch.

\subsection{Analiza ryzyka}

\subsubsection{Czas ewaluacji i eksperymentów}
Stworzenie biblioteki wiąże się z intensywnym treningiem oraz ewaluacją
generowanych sieci. Trening wielogłowicowej sieci neuronowej dla wielu autorów, dający zadowalające 
wyniki, zajmuje dużo czasu. Może być on rozpoczęty dopiero po przetworzeniu danych wejściowych
oraz poprawnej implementacji modelu sieci. W przypadku wydłużenia się implementacji problemem może być
odpowiednie wytrenowanie sieci ze względu na ograniczenie czasowe związane z oddaniem pracy inżynierskiej.
Do przyspieszenia czasu wykonywania obliczeń wykorzystany zostanie klaster obliczeniowy Prometheus.

\subsubsection{Wiedza na temat sieci neuronowych}
Sieci neuronowe to złożona gałąź sztucznej iteligencji, która wymaga sprawnego posługiwania się formalizmem matematycznym.
Jest to dla nas największa przeszkoda. Opanowanie tej wiedzy wymaga czasu oraz dużej cierpliwości, 
szczególnie, że jest to dla nas nowe zagadnienie.

\subsubsection{Znajomość języków programowania oraz bibliotek }
Przeszkodą może być opanowanie biblioteki PyTorch, która jest dla nas nowością.

\subsubsection{Aspekt ekonomiczny }
Będziemy korzystać z darmowych bibliotek oraz środowiska, na które mamy licencje jako studenci.