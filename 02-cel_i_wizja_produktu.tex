%##noBuild
\newpage
\section{Cel i wizja produktu}

\subsection{Definicja problemu}
Sieci neuronowe pozwalają nam sprawnie ropoznawać i klasyfikować pewne wzorce. Można je zastosować, aby
mając do dyspozycji zestaw tekstów znanych autorów oraz tekst o nieznanym pochodzeniu stwierdzić, któremu z nich 
przypisać jego autorstwo. W celu identyfikacji, należy porównać model języka stosowanego w tekście
ze stosowanym przez znanych autorów.
\newline
\newline
Problem ten był tematem zadania konkursowego PAN 2015 \cite{pan}.
Najskuteczniejszym rozwiązaniem okazało się użycie wielogłowicowych, rekurencyjnych sieci 
neuronowych Douglasa Bagnalla \cite{bagnall}, dlatego zdecydowaliśmy się na ich użycie.

 
\subsection{Cel pracy}
Celem projektu jest stworzenie biblioteki do tworzenia i trenowania sieci dla zadania 
identyfikacji autorstwa tekstów. Praca ta jest rozwinięciem innej pracy inżynierskiej z poprzednich lat \cite{radzio}.
Osiągnięto w niej dokładność na poziomie 85\% dla języka holenderskiego. Dla  języka angielskiego, greckiego
oraz hiszpańskiego dokładność oscylowała w okolicach 50\%. Celem pracy jest poprawa tych wyników.


\subsection{Motywacja}
Mimo pojawienia się wielu zaawansowanych narzędzi do łatwego tworzenia nawet bardzo skomplikowanych sieci, 
zapewniających przy tym naukę oraz kontrolę nad jej przebiegiem wciąż brakuje biblioteki dla zadania identyfikacji autorstwa 
tekstu. Internet pozwala na wysyłanie wiadomości z poczuciem anonimowości, co może być wykorzystane do pisania
gróźb lub do innych celów łamiących prawo. Istnieją techniki pozwalające nam na stwierdzenie z jakiego adresu IP nadana
została wiadomość, nie zawsze jednak dzięki temu jesteśmy w stanie poznać dokładną tożsamość osoby stojącej za groźbami.
Z pomocą mogła by przyjść nasza biblioteka, która na podstawie różnych komentarzy internetowych pomogła
by zidentyfikować przestępce.
Innym zastosowaniem wykonanej przez nas biblioteki może być idenftyfikacja autorstwa tekstów historycznych, których autorstwo jest
nieznane, na podstawie porównania ze stylami językowymi osób piszących w tamtym okresie czasu.
Biblioteka weryfikująca autorstwo może posłużyć jako podstawa systemu antyplagiatowego. Systemy takie, 
jak na przykład Jednolity System Antyplagiatowy (JSA) używane są na wszystkich uczelniach w kraju.
Niedopuszczalne jest aby student otrzymał tytuł na podstawie pracy innej osoby

\subsection{Wizja}
Wizją tej pracy jest stworzenie dopracowanej biblioteki pozwalającej na konfigurowalne tworzenie 
głębokich, rekurencyjnych sieci neuronowych (RNN), ich trening, oraz preprocesing danych wejściowych. Oprócz tego powinna ona 
zapewnić persystencję wytrenowanych modeli oraz ich dalsze testowanie. Biblioteka powinna wspierać 
naukę w czterech językach: angielskim, greckim, hiszpańskim oraz holenderskim. Wyżej wspomniana konfigurowalność powinna 
pozwalać na tworzenie dowolnych sieci parametryzowanych hiperparametrami: głębokość sieci, wymiar 
wektorów wejściowych, liczba neuronów w warstwach ukrytych, wielkość mini-batchów, itd.
Biblioteka powstanie w języku Python przy wykorzystaniu biblioteki PyTorch.

\subsection{Analiza ryzyka}

\subsubsection{Czas ewaluacji i eksperymentów}
Stworzenie biblioteki wiąże się z intensywnym treningiem oraz ewaluacją
generowanych sieci. Trening wielogłowicowej sieci neuronowej dla wielu autorów, dający zadowalające 
wyniki, zajmuje dużo czasu. Może być on rozpoczęty dopiero po przetworzeniu danych wejściowych
oraz poprawnej implementacji modelu sieci. W przypadku wydłużenia się implementacji problemem może być
odpowiednie wytrenowanie sieci ze względu na ograniczenie czasowe związane z oddaniem pracy inżynierskiej.
Do przyspieszenia czasu wykonywania obliczeń wykorzystany zostanie klaster obliczeniowy Prometheus.

\subsubsection{Wiedza na temat sieci neuronowych}
Sieci neuronowe to złożona gałąź sztucznej iteligencji, która wymaga sprawnego posługiwania się formalizmem matematycznym.
Jest to dla nas największa przeszkoda. Opanowanie tej wiedzy wymaga czasu oraz dużej cierpliwości, 
szczególnie, że jest to dla nas nowe zagadnienie.

\subsubsection{Znajomość języków programowania oraz bibliotek }
Przeszkodą może być opanowanie biblioteki PyTorch, która jest dla nas nowością.

\subsubsection{Aspekt ekonomiczny }
Będziemy korzystać z darmowych bibliotek oraz środowiska, na które mamy licencje jako studenci.