\newpage
\section{Preprocessing}

Preprocessing to moduł naszej biblioteki odpowiedzialny za przetwarzanie tekstów na format
nadający się do szkolenia sieci. 

\subsection{Teoria}

Proces preprocessingu dzieli się na dwa etapy.
\begin{enumerate}
	\item Redukcja alfabetu (mapowanie znaków)
	\item Zamiana zredukowanych tekstów na tensory
\end{enumerate}

Należy przy tym dodać że domyślnie dostępnymi językami w bibliotece są:
\languages 


\subsubsection{Redukcja alfabetu}
Redukcja alfabetu to sprytny sposób zmniejszenia ilości znaków w alfabecie, tak by uczenie sieci było
efektywniejsze. Możemy definiować nasze własne 'mappery'. Przykład pliku do redukcji alfabetu języka
angielskiego znajduje się poniżej (został on znacznie skrócony wzlgędem oryginału żeby zachować lepszą
czytelność)

\begin{python}
charmap = {
    u' ': u' ',                 # kept Zs (92447)
    u'e': u'e',
    u't': u't',
    u'o': u'o',
    u'a': u'a',
    u'n': u'n',
    u'i': u'i',
    u's': u's',
    u'h': u'h',
    u'r': u'r',
    u'\n': u'\n',               # kept Cc (13082)
    u'l': u'l',
    u'd': u'd',
    u'u': u'u',
    u'y': u'y',
    u'm': u'm',
    u'g': u'g',
    u'w': u'w',
    u',': u',',                 # kept Po (5250)
    u"'": "'",                  # single quote
    u'.': u'.',                 # kept Po (4826)
    u'f': u'f',
    u'c': u'c',
    u'b': u'b',
    u'p': u'p',
    u'-': u'-',                 # kept Pd (3253)
    u'k': u'k',
    u'I': u'\xb9i',             # decomposed caps
    u'v': u'v',
}
\end{python}

\newpage
\subsubsection{Zamiana tekstów na tensory}
Aby zamienić teksty na tensory potrzebny jest nam alfabet. Alfabet definiujemy jako listę, której 
elementy są znakami. Tak wygląda fragment alfabetu z języka angielskiego:
\begin{python}
alphabet = [
    ' ',
    'e',
    't',
    'o',
    'a',
    'n',
    'i',
    's',
    'h',
    'r',  # 10
    '\\n',
    'l',
    'd',
    'u',
    'y',
    'm',
    'g',
    'w',
    ',',
    "'",  # 20
    '.',
    'f',
    'c',
    'b',
]
\end{python}

Każdy znak zredukowanego pliku wejściowego zamieniany jest na wektor wykorzystując alfabet, 
a wektory 'umieszczane' są w tensorze. Zamianę liter na wektory najłatwiej pokazać na przykładzie.

Załóżmy że nasz alfabet wygląda następująco:

\begin{python}
alphabet = [
    'a',  #0
    'b',  #1
    'c',  #2
    'd',  #3
 ]
\end{python} 
Jeśli zamienimy literę 'a' na wektor otrzymamy: 
 
\vspace{2mm}
$
\begin{bmatrix} 
1, & 0, & 0, & 0
\end{bmatrix} 
$
\vspace{2mm}

Ponieważ litera 'a' jest na pozycji 0 w alfabecie liczba 1 pojawiła się na pozycji 0 w wektorze.
Długość wektora wynosi 4 ponieważ tyle znaków znajduje się w alfabecie. A więc dla litery 
'c' wektor wyglądałby w następujący sposób:

\vspace{2mm}
$
\begin{bmatrix} 
0, & 0, & 1, & 0
\end{bmatrix} 
$
\vspace{2mm}

W ten sposób konwertowane są wszystkie znaki w plikach, które następnie umieszczane są w tensorach.

TODO
TU OPISAĆ BUDOWE TENSORÓW

\newpage
\subsubsection{Rozdzielenie tensorów na paczki}
Jednym ze sposobów przyspieszenia czasu uczenia jest podzielenie danych na paczki (ang. batche).
Pozwala nam to na wykonywanie obliczeń na wielu danych jednocześnie. Po przygotowaniu tensora wypełnionego wektorami
wysyłamy go do BatchProcessora, którzy dzieli nasze dane na paczki o określonym rozmiarze. 

\vspace{2mm}
Tak przygotowane zestawy danych możemy ładować do sieci.

\subsection{Pliki wejściowe oraz wyjściowe}
Aby korzystać z modułu preprocessingu należy zachować określoną konwencję danych wejściowych. 

\subsubsection{Struktura wejścia}

\begin{enumerate}
	\item Wszystkie pliki tekstowe muszą być w formacie txt,
	\item folder z danymi wejściwoymi musi mieć konkretną strukture. W folderze każdego autora
		  powinny znajdować się jego teksty: 
			\begin{itemize}
				\item knownX.txt gdzie X jest numerem tekstu, tekstów known.txt może być wiele.
				\item unknown.txt
			\end{itemize}
\end{enumerate}

Przykład: 

\myspace
\dirtree{%
.1 data.
.2 authors.
.3 EN001.
.4 known01.txt.
.4 unknown.txt.
.3 EN002.
.4 known01.txt.
.4 known02.txt.
.4 known03.txt.
.4 unknown.txt.
.3 EN003.
.4 known01.txt.
.4 unknown.txt.
.3 EN004.
.4 known01.txt.
.4 known01.txt.
.4 unknown.txt.
.3 EN005.
.4 known01.txt.
.4 unknown.txt.
}
\myspace

\newpage
\subsubsection{Struktura wyjścia}

Wszystkie pliki wejściowe są zamieniane na tensory i dzielone na batche. Wszystkie pliki knownX.txt 
danego autora są wcześniej łączone w jeden plik, który póniej staje się tensorem. W ten sposób 
otrzymujemy następująco wyglądający folder wyjściowy, gdzie każdy plik ENX.pt (gdzie X to numer)
 jest tensorem:

\myspace
\dirtree{%
.1 tensors.
.2 known.
.3 EN001.
.4 EN001.pt.
.3 EN002.
.4 EN002.pt.
.3 EN003.
.4 EN003.pt.
.3 EN004.
.4 EN004.pt.
.3 EN005.
.4 EN005.pt.
.2 unknown.
.3 EN001.
.4 EN001.pt.
.3 EN002.
.4 EN002.pt.
.3 EN003.
.4 EN003.pt.
.3 EN004.
.4 EN004.pt.
.3 EN005.
.4 EN005.pt.
}
\myspace


\newpage
\subsection{Jak korzystać?}
Moduł preprocessingu został stworzony z myślą o prostej i intuicyjnej obsłudze. Wykorzystane przy
tym zostały Pythone'owe słowniki które pozwalają na uproszczenie sygnatury funkcji o parametry, które
w danym momencie nie są wykorzystywane. Najprościej zasadę działania modułu preprocessingu można 
pokazać na przykładach. Folder wejściowy z danymi musi być o identycznej strukturze jak ten na 
konkursie PAN.
Załóżmy że nasz katalog projektu wygląda w następujący sposób:




\myspace
\dirtree{%
.1 src.
.2 main.py.
.1 data.
.2 authors.
.3 EN001.
.4 known01.txt.
.4 unknown.txt.
.3 EN002.
.4 known01.txt.
.4 known02.txt.
.4 known03.txt.
.4 unknown.txt.
.3 EN003.
.4 known01.txt.
.4 unknown.txt.
.3 EN004.
.4 known01.txt.
.4 known01.txt.
.4 unknown.txt.
.3 EN005.
.4 known01.txt.
.4 unknown.txt.
}
\myspace

Kiedy posiadamy już odpowiednią strukture katalogów z plikami wejściowymi możemy stworzyć 
odpowiedni obiekt i przystąpić do preprocessingu. Zakładamy że wszystkie linijki przedstawianego 
tutaj kodu znajdują się w widocznym na diagramie struktury folderów pliku main.py.

\begin{python}
from library.preprocessing.preprocessing import Preprocessing

Preprocessing()

\end{python}

Parametry konstruktora modułu preprocessingu będą określać w jaki sposób ma funkcjonować. 
oto wszystkie możliwe parametry:
\begin{itemize}
	\item language
	\item data\_path
	\item tensors\_path
	\item mapped\_save\_path
	\item mapped\_source\_path
	\item batch\_size
	\item log\_lvl
	\item only\_reduce
	\item check\_files
	\item alphabet
	\item mapper
\end{itemize}

Parametry możemy używać w dowolnej kombinacji, jednak nie wszystkie z nich będą miały sens. Zbędne
parametry będą ignorowane.

\subsubsection{language}
Parametr language jest parametrem obowiązkowym i pozwala nam określić język tekstów wejściowych.
\begin{python}
from library.preprocessing.preprocessing import Preprocessing

Preprocessing(language="en")

\end{python}

Jeśli go nie zdefiniujemy, powinniśmy otrzymać następujący błąd:

\begin{consolerror}
You have to specify language | EXAMPLE: (language="en")
\end{consolerror}

Domyślnie dostępnymi językami w bibliotece są:

\languages 

\subsubsection{data\_path}
Parametr data\_path pozwala nam zdefiniować ścieżkę do danych wejściowych. W naszym przypadku
będzie ona wyglądała w następujący sposób:
\begin{python}
from library.preprocessing.preprocessing import Preprocessing

Preprocessing(language="en",
              data_path="../data/authors/")
                   
\end{python}

Jeśli jej nie zdefiniujemy, powinniśmy otrzymać następujący błąd:

\begin{consolerror}
You have to specify data source | EXAMPLE: (data_path=/some/path)
\end{consolerror}

Przy tak zdefiniowanych parametrach wszystkie operacje zostaną wykonane w ramie, a output w postaci
tensorów zostanie zapisany w katalogu pliku, który wykonujemy. Katalog z tensorami nosi nazwę tensors.
Struktura katalogów po wykonaniu będzie wyglądać następująco:

\myspace
\dirtree{%
.1 src.
.2 main.py.
.2 tensors \DTcomment{katalog wyjściowy z tensorami}.
.1 data.
.2 authors.
}
\myspace

\newpage
\subsubsection{tensors\_path}
Jeśli chcemy zdefiniować ścieżkę zapisu tensorów możemy tego dokonać używając parametru 
tensors\_path. Dla przykładu zdefiniuję ją w następujący sposób. 
\begin{python}
from library.preprocessing.preprocessing import Preprocessing

Preprocessing(language="en",
              data_path="../data/authors/",
              tensors_path="../data/tensors/")
                   
\end{python}

Po wykonaniu pliku main.py otrzymamy następującą strukture katalogów:
\myspace
\dirtree{%
.1 src.
.2 main.py.
.1 data.
.2 authors.
.2 tensors \DTcomment{katalog wyjściowy z tensorami}.
}
\myspace

Wszystkie operacje wykonywane są w tym przypadku przy użyciu pamięci ram.


\subsubsection{mapped\_save\_path}
Mamy równiez możliwość zapisania plików o zredukowanym alfabecie jeśli chcielibyśmy je przejrzeć lub
wykorzystać w przyszłości ponownie. Do tego służy opisywany właśnie parametr mapped\_save\_path.
UWAGA! Za każdym razem kiedy korzystamy z tego argumentu wszystkie pliki spod wskazanej ścieżki są 
usuwane.

\begin{python}
from library.preprocessing.preprocessing import Preprocessing

Preprocessing(language="en",
              data_path="../data/authors/",
              tensors_path="../data/tensors/"
              mapped_save_path="../data/reduced_authors/")
                   
\end{python}
Stuktura zredukowanego katalogu z tekstami jest identyczna jak struktura katalogu wejściowego. 

Po wykonaniu pliku main.py otrzymamy następującą strukture katalogów:

\myspace
\dirtree{%
.1 src.
.2 main.py.
.1 data.
.2 authors.
.2 tensors \DTcomment{katalog wyjściowy z tensorami}.
.2 reduced\_authors \DTcomment{katalog ze zredukowanymi tekstami}.
}
\myspace

Wszystkie operacje wykonywane są w tym przypadku przy użyciu pamięci ram.

\newpage
\subsubsection{mapped\_source\_path}
Jeśli chcielibyśmy skorzystać ze zredukowanych wcześniej tekstów, które mamy zapisane na dysku 
możemy w tym celu użyć parametru mapped\_source\_path.

Załóżmy przy tym następujące drzewo katalogów:
\myspace
\dirtree{%
.1 src.
.2 main.py.
.1 data.
.2 reduced\_authors \DTcomment{katalog ze zredukowanymi tekstami}.
}
\myspace

\begin{python}
from library.preprocessing.preprocessing import Preprocessing

Preprocessing(language="en",
              tensors_path="../data/tensors/",
              mapped_source_path="../data/reduced_authors/")
                   
\end{python}

Po wykonaniu pliku main.py wyjściowe drzewo katalogów powinno wyglądać w następujący sposób:
\myspace
\dirtree{%
.1 src.
.2 main.py.
.1 data.
.2 tensors \DTcomment{katalog wyjściowy z tensorami}.
.2 reduced\_authors \DTcomment{katalog ze zredukowanymi tekstami}.
}
\myspace

Pliki zreduykowane zostały w tym przypadku odczytane z dysku, a nie z pamięci ram jak miało to miejsce
we wcześniejszych przykładach. 
\newline

Możemy jednak jeszcze bardziej mieszać parametrami. Załóżmy że nie chcemy aby pliki redukowane były
do pamięci ram, a następnie w ten sposób przekazywane do tworzenia tensorów. Chcemy aby pliki najpierw
zredukowały się na dysk, a następnie z dysku zostały zamienione na tensory. 
Załóżmy wejściowe drzewo katalogów:

\myspace
\dirtree{%
.1 src.
.2 main.py.
.1 data.
.2 authors.
}
\myspace

Kod wyglądałby następująco:
\begin{python}
from library.preprocessing.preprocessing import Preprocessing

Preprocessing(language="en",
              data_path="../data/authors/",
              tensors_path="../data/tensors/",
              mapped_save_path="../data/reduced_authors/",
              mapped_source_path="../data/reduced_authors/")

\end{python}

Parametr data\_path ma wyższy priorytet niż parametr mapped\_source\_path, a więc biblioteka najpierw 
pobierze niezredukowanych autorów z katalogu authors, następnie zredukuje ich do katalogu reduced\_authors,
następnie wykorzystując katalog reduced\_authors stworzy tensory do katalogu tensors.

Wyjściowe drzewo katalogów będzie wyglądać następująco:

\myspace
\dirtree{%
.1 src.
.2 main.py.
.1 data.
.2 authors.
.2 reduced\_authors \DTcomment{katalog ze zredukowanymi tekstami}.
.2 tensors \DTcomment{katalog wyjściowy z tensorami}.
}

\myspace

\subsubsection{only\_reduce}
Możemy także skorzystać tylko z redukcji, bez tworzenia tensorów. Służy do tego argument only\_reduce.
Aby nastąpiła jedynie redukcja alfabetu należy ustawić go na True jak w kodzie poniżej:

\begin{python}
from library.preprocessing.preprocessing import Preprocessing

Preprocessing(language='en',
              data_path="../data/authors/",
              mapped_save_path = "../data/reduced_authors/",
              reduce_only=True) 

\end{python}

\subsubsection{batch\_size}
Możemy również zdefiniować rozmiar paczek na jakie podzielony zostanie nasz tensor. Dokonać tego możemy poprzez
parametr batch\_size. Domyślna wartość tego parametru to 20.

\begin{python}
language = 'en'
data_path = "../data/authors/"
tensors_path = "../data/tensors/"
mapped_save_path = "../data/reduced_authors/"
Preprocessing(language=language,
              data_path=data_path,
              tensors_path=tensors_path,
              mapped_save_path=mapped_save_path,
              batch_size=30)

\end{python}


W takim wypadku dane zostaną podzielone na paczki o rozmiarze 30.
\subsubsection{log\_lvl}

\TODO{opisac - możliwość logowania na jednym z 3 poziomów} 

\subsubsection{check\_files}

\TODO{opisac - przechodzenie po katalogach i sprawdzanie nazw plików (czy np. użytkownik nie zapomniał
zmienić nazwy jakiegoś pliku)}

\subsubsection{mapper}

\TODO{opisac - własny mapper do redukcji alfabetu}


\subsubsection{alphabet}

\TODO{opisac - własny alfabet}


