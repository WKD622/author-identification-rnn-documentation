\section{Wstęp}
 
\subsection{Autorzy}
\begin{itemize}
\item Jan Liberacki
\item Jakub Ziarko
\end{itemize}

\subsection{Wprowadzenie}
\TODO{uzupełnić}

\newpage
\subsection{Jak czytać dokumentację?}


\begin{itemize}
\item Każdy kod w Pythonie formatowany jest w następujący sposób:

\begin{python}
# Get three test score
round1 = int(raw_input("Enter score for round 1: "))

round2 = int(raw_input("Enter score for round 2: "))

round3 = int(raw_input("Enter score for round 3: "))
   
# Calculate the average
average = (round1 + round2 + round3) / 3

# Print out the test score
print "the average score is: ", average 

\end{python}

\item Tak oznaczane są samotnie funkcjonujące importy / ścieżki do plików
\begin{import}
from library.preprocessing.preprocessing import Preprocessing
\end{import}

\item Kod w bashu lub output z terminala formatowany jest w następujący sposób:

\begin{bash}
#!/bin/bash
echo "Hello World"
\end{bash}

\item Natomiast tak formatowany jest błąd pokazujący się w terminalu:
\begin{consolerror}
You have to specify language | EXAMPLE: (language="en")
\end{consolerror}



\item W ten sposób przedstawiana jest struktura katalogów:
\myspace
\dirtree{%
.1 src.
.2 main.py.
.1 data.
.2 file1.txt.
.2 file2.txt.
}

\end{itemize}