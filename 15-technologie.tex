\section{Technologie}

\subsection{Python}
Jest to język programowania wysokiego poziomu ogólnego przeznaczenia.
Jego składnia cechuje się przejrzystością i zwięzłością. Python wspiera pisanie w sposób obiektowy, 
imperatywny a także funkcyjny. Jest językiem interpretowanym i dynamicznie typowanym. Jest bardzo popluarnym 
wyborem w kontekście problemów uczenia maszynowego - posiada pod tym względem szeroki wybór bibliotek, 
dużą społeczność, a zatem i sporo materiałów w sieci.
Początkowo może się wydawać, że nie jest to dobry wybór w kwestii wydajności, 
są jednak biblioteki jak Numpy, które praktycznie go rozwiązują.

\subsection{PyTorch}
Jest to biblioteka dla języka Python przeznaczona do uczenia maszynowego. PyTorch został utworzony
przez oddział sztucznej inteligencji Facebooka. Cechuje go przejrzysty interface, duża społeczność
oraz duża ilość materiałów w sieci. Była to jedyna biblioteka, która została użyta w kontekście 
stworzenia naszej sieci neuronowej. Posiada wsparcie dla GPU.

\subsection{Bash}
Jest to język skryptowy powłoki systemowej Linuxa wykorzystany przez nas do pisania krótkich skryptów
do operacji na plikach, jak na przykład transfer plików z i na Prometheusa.
