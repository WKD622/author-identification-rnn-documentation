\section{Podsumowanie}

Produkt, który stworzyliśmy spełnia założenia dotyczące trenowania sieci dla zadania weryfikacji autorstwa.
Niestety wyniki końcowe nie są satysfakcjonujące z powodu niskiej skuteczności. Problemy implementacyjne 
napotkane po drodze mocno skróciły nasz czas na odpowiednie dopracowanie wyników końcowych.
\newline
\newline
Niniejsza praca pozwoliła nam bardziej zgłębić się w tematykę sieci neuronowych, 
a także zapoznać się z nowymi technologiami, w których nie mieliśmy wcześniej zbyt wiele doświadczenia
(np. biblioteka \texttt{Pytorch}). Projekt wpłynął pozytywnie również na nasze zdolności organizacyjne
odnośnie organizacji czasu i podziału obowiązków.

\subsection{Spotkania}
Spotkania naszego zespołu występowały w częstotliwości 1-2 razy w tygodniu. W szczególności na etapie
budowania modelu sieci oraz trenowania, gdzie kod pisany był wspólnie.
Komunikowaliśmy się pomiędzy sobą również na bieżąco za pośrednictwem komunikatorów internetowych, 
raportując aktualny postęp prac i najbliższe plany.

Kilka razy w semestrze spotykaliśmy się w ramach Pracowni Projektowej z jej prowadzącym dr hab. inż. Rafałem Dreżewskim.
W trakcie zajęć prezentowaliśmy aktualny stan pracy oraz dyskutowaliśmy na temat planów na najbliższy okres czasu.
\newline
\newline
Wielokrotnie w czasie semestru organizowaliśmy również spotkania z naszym promotorem dr inż. Marcinem Kutą 
w celu prezentacji zakończonego właśnie cyklu tworzenia dokumentacji oraz dyskusji nad problemami jakie napotkaliśmy.

\subsection{Podziękowania}
Pragniemy serdecznie podziękowac dr. inż. Marcinowi Kucie za opiekę nad naszym projektem.
Dziękujemy również za dostęp do infrastruktury PL-Grid, gdzie na klastrze Prometeusz trenowana była
nasza sieć.