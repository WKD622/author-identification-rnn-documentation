\newpage
\section{Jak czytać dokumentację?}

Wyróżniamy następujące formatowania:

\begin{itemize}
\item Kod w Pythonie

\begin{python}
# Get three test score
round1 = int(raw_input("Enter score for round 1: "))

round2 = int(raw_input("Enter score for round 2: "))

round3 = int(raw_input("Enter score for round 3: "))
   
# Calculate the average
average = (round1 + round2 + round3) / 3

# Print out the test score
print "the average score is: ", average 

\end{python}

\item Importy lub ścieżki do plików:
\begin{import}
from library.preprocessing.preprocessing import Preprocessing
\end{import}

\item Kod w bashu:

\begin{bash}
#!/bin/bash
echo "Hello World"
\end{bash}

\item Błąd pokazujący się w terminalu:
\begin{consolerror}
You have to specify language | EXAMPLE: (language="en")
\end{consolerror}



\item Struktura katalogów:
\myspace
\dirtree{%
.1 src.
.2 main.py.
.1 data.
.2 file1.txt.
.2 file2.txt.
}

\end{itemize}