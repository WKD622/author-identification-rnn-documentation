\newpage
\subsection{Wykorzystanie modułu do trenowania sieci}

\myspace
\dirtree{%
.1 network.
.2 train <----.
}
\myspace

\begin{itemize}

\item {Train [ klasa ] }
\begin{import}
from library.network.train import Train

rnn = Train(hidden_size=100,
            num_layers=1,
            num_epochs=1,
            batch_size=40,
            timesteps=50,
            learning_rate=0.004,
            authors_size=100,
            vocab_size=48,
            save_path='results',
            tensors_path='data/tensors',
            language='DU',
            truth_file_path='data/dutch/truth.txt')
\end{import}

Przyjmuje następujące argumenty:
\begin{itemize}
	\item hidden\_size: int
		\newline Wartość oznaczająca rozmiar warstwy ukrytej.
	\item num\_layers: int
		\newline Ilość warstw ukrytych.
	\item num\_epochs: int
		\newline Ilość epok.
	\item batch\_size: int
		\newline Wielkość paczek danych wprowadzanych do sieci.
	\item timesteps: int
		\newline Parametr określający liczbę timesteps.
	\item learning\_rate: float
		\newline Współczynnik uczenia.
	\item authors\_size: int
		\newline Liczba różnych autorów.
	\item vocab\_size: int
		\newline Wielkość alfabetu.
	\item save\_path: str
		\newline Ścieżka zapisu modeli.
	\item tensors\_path: str
		\newline Ścieżka do tensorów autorów po preprocessingu.
	\item language: str
		\newline Język w jakim są teksty.
		
\end{itemize}

Po przekazaniu odpowiednich parametrów wystarczy jedynie wywołać metodę train() na obiekcie rnn, 
która rozpocznie proces trenowania.