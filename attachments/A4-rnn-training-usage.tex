\newpage
\subsection{Wykorzystanie modułu do trenowania sieci}
\TODO{jakis tekst}
\myspace
\dirtree{%
.1 network.
.2 train <----.
}
\myspace

\item{Train [ klasa ]}
\begin{import}
from library.network.train import Train

rnn = Train(hidden_size=100,
            num_layers=1,
            num_epochs=1,
            batch_size=40,
            timesteps=50,
            learning_rate=0.004,
            authors_size=100,
            vocab_size=48,
            save_path='results',
            tensors_path='data/tensors',
            language='DU',
            truth_file_path='data/dutch/truth.txt')
            
rnn.train()
\end{import}

Przyjmuje następujące argumenty:
\begin{itemize}
	\item hidden\_size: int
		\newline Wartość oznaczająca rozmiar warstwy ukrytej.
	\item num\_layers: int
		\newline Liczba warstw ukrytych.
	\item num\_epochs: int
		\newline Liczba epok.
	\item batch\_size: int
		\newline Wielkość paczek danych wprowadzanych do sieci.
	\item timesteps: int
		\newline Parametr określający liczbę timesteps.
	\item learning\_rate: float
		\newline Współczynnik uczenia.
	\item authors\_size: int
		\newline Liczba różnych autorów.
	\item vocab\_size: int
		\newline Wielkość alfabetu.
	\item save\_path: str
		\newline Ścieżka zapisu modeli.
	\item tensors\_path: str
		\newline Ścieżka do tensorów autorów po preprocesingu.
	\item language: str
		\newline Język w jakim są teksty.
		
\end{itemize}

\begin{itemize}

  \item{train}
  \newline Najważniejsza metoda tej klasy. Odpowiada za cały proces trenowania oraz ewaluacji sieci. Po 
  odpowiednim zainicjalizowaniu obiektu rekurencyjnej sieci neuronowej wystarczy jedyne wywołać tę metodę.
  \item{get\_accuracy}
  \item \TODO{batch_processor czy wystepuje}
  \newline Metoda odpowiedzialna za proces ewaluacji, a dokładniej badania skuteczności sieci. Korzysta z 
  oddzielnego batch-procesora, stworzonego na podstawie zbioru walidacyjnego. Zwraca wyniki w zakresie
  [0,1], gdzie 1 oznacza stu procentową skuteczność a 0 brak poprawnych przyporządkowań.
  \item{OutputManager [klasa]}
  \newline Wykorzystywana do sprawnego i przejrzystego logowania aktualnych wyników. Informuje nas o aktualnej
  skuteczności, wartości funkcji kosztu czy też czasie uczenia. Tak spreparowane wyniki zapisuje również do pliku.

\end{itemize}