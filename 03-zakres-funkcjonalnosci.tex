%##noBuild
\newpage
\section{Zakres funkcjonalności}

\subsection{Charakterystyka użytkowników systemu}
Użytkownik biblioteki wykorzystuje dostarczone narzędzia do tworzenia i trenowania sieci 
dla zadania weryfikacji autorstwa tekstu. W celu dokładniejszego dopasowania do potrzeb, 
użytkownik dobiera optymalne dla siebie parametry dostarczone przez bibliotekę.

 
\subsection{Wymagania funkcjonalne}
Finalna wersja biblioteki powinna spełniać poniższe wymagania funkcjonalne:
\begin{itemize}
 	\item Dla modułu preprocessingu:
	    \begin{itemize}
		    \item Dostarczenie mechanizmu redukcji alfabetu.
		    \item Przygotowanie danych poprzez zamianę tekstów na tensory oraz możliwość 
				zapisania tak przygotowanych danych w celu ponownego użycia bez potrzeby
				powtórzonego preprocessingu. 
	    \end{itemize}
 	\item Dla modułu trenowania sieci:
		\begin{itemize}
			\item Możliwość doboru parametrów modelu sieci takich jak:
				\begin{itemize}
			    	\item Rozmiar warstwy ukrytej,
			    	\item Głębokość sieci,
			    	\item Maksymalna liczba różnych autorów tekstów w modelu,
		    	\end{itemize}
			\item Parametryzowania trenowania sieci dobierając:
				\begin{itemize}
			    	\item Rozmiar mini-batchów danych,
			    	\item Współczynnik uczenia,
			    	\item Tempo zmniejszania się współczynnika uczenia po przetworzeniu pojedynczej epoki,
			    	\item Maksymalną liczbę epok,
			    	\item Długość sekwencji znaków, na której uczony będzie model,
		    	\end{itemize}
		  \end{itemize}
\end{itemize}
Dodatkowo biblioteka dostarczać będzie moduł logujący aktualny progres uczenia 
oraz finalne wyniki a także ich dokładność.	

\newpage
\subsection{Wymagania niefunkcjonalne}
\begin{itemize}
 	\item Korzystanie z jezyka Python wraz z biblioteka PyTorch.
 	\item Wykorzystanie systemu kontroli wersji Git do przechowywania kodu oraz dokumentacji.
 	\item Użycie Korpusu tekstów opublikowanego na potrzeby PAN 2015 do trenowania i ewaluacji sieci.
 	\item Trenowanie sieci wykorzystując grant na zasoby obliczeniowe ACK Cyfronet AGH.
\end{itemize}
