%##noBuild
\newpage
\section{Zakres funkcjonalności}

\subsection{Charakterystyka użytkowników systemu}
Biblioteka ma za zadanie być przystępna dla osób doświadczonych w zakresie sieci neuronowych jak również
dla użytkowników nie zaznajomionych z tym tematem. Dlatego też użytkowników systemu możemy podzielić 
na dwie grupy:
\begin{itemize}
 	\item Zaawansowany użytkownik - Osoba doświadczona w zakresie sieci neuronowych. Użytkownik korzysta
 	z biblioteki dla zadania weryfikacji autorstwa tekstu, może roszerzyć funkcjonalność o własny rodzaj
 	mapowania alfabetu, wykorzystać porządaną przez siebie funkcje kosztu. Może być to student pragnący
 	wykorzystać bibliotekę do własnego projektu odpowiednio ją dostosowując do swoich potrzeb.
 	\item Początkujący użytkownik - Osoba nie zainteresowana tematem sieci neuronowych. Korzysta z biblioteki
 	wykonując podstawowe operacje opisane w podręczniku użytkowania, a także korzysta z zalecanych parametrów.
 	Może być to pracownik uczelni pragnący zweryfikować autorstwo pracy inżynierskiej.
\end{itemize}

 
\subsection{Wymagania funkcjonalne}
Finalna wersja biblioteki powinna spełniać poniższe wymagania funkcjonalne:
\begin{itemize}
 	\item Dla modułu preprocesingu:
	    \begin{itemize}
		    \item Dostarczenie mechanizmu redukcji alfabetu.
		    \item Przygotowanie danych poprzez zamianę tekstów na tensory oraz możliwość 
				zapisania tak przygotowanych danych w celu ponownego użycia, bez potrzeby
				powtórzonego preprocesingu. 
	    \end{itemize}
 	\item Dla modułu trenowania sieci:
		\begin{itemize}
			\item Możliwość doboru parametrów modelu sieci takich jak:
				\begin{itemize}
			    	\item Rozmiar warstwy ukrytej,
			    	\item Głębokość sieci,
			    	\item Maksymalna liczba różnych autorów tekstów w modelu,
		    	\end{itemize}
			\item Parametryzowania trenowania sieci dobierając:
				\begin{itemize}
			    	\item Rozmiar mini-batchów danych,
			    	\item Maksymalną liczbę epok,
			    	\item Długość sekwencji znaków, na której uczony będzie model,
		    	\end{itemize}
		  \end{itemize}
\end{itemize}
Dodatkowo biblioteka dostarczać będzie moduł logujący aktualny progres uczenia 
oraz finalne wyniki, w tym dokładność identyfikacji autorstwa.	

\subsection{Wymagania niefunkcjonalne}
\begin{itemize}
 	\item Wykorzystanie języka Python wraz z biblioteka PyTorch \cite{pytorch}.
 	\item Wykorzystanie systemu kontroli wersji Git do przechowywania kodu oraz dokumentacji.
 	\item Użycie korpusu tekstów opublikowanego na potrzeby PAN 2015 \cite{pan} do trenowania i ewaluacji sieci.
 	\item Trenowanie sieci z wykorzystaniem zasobów obliczeniowych ACK Cyfronet AGH \cite{plgrid}.
\end{itemize}
