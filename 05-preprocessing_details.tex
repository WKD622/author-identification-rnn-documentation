\newpage
\subsection{Budowa}
W tej części opisana zostanie budowa modułu Preprocessingu - z jakich fragmentów się składa
oraz za co one odpowiadają. Opisane zostaną tylko ważniejsze fragmenty kodu z racji na jego 
obszerność.

Budowa modułu preprocessingu:


\myspace
\dirtree{%
.1 preprocessing.
.2 batch\_processing \DTcomment{metody niskopoziome odpowiedzialne za podział na batche}.
.2 chars\_mapping \DTcomment{metody niskopoziome odpowiedzialne za redukcje alfabetów}.
.2 data\_structs	\DTcomment{klasy do trzymania tekstów w formie obiektów}.
.2 files \DTcomment{metody niskopoziome odpowiedzialne za operacje na plikach}.
.2 to\_tensor \DTcomment{metody niskopoziome odpowiedzialne za zamianę tekstu na tensory}.
.2 constants.py	\DTcomment{wszystkie stałe}.
.2 exceptions.py \DTcomment{wszystkie wyjątki}.
.2 preprocessing.py \DTcomment{główny plik modułu}.
}
\myspace

\subsubsection{preprocessing.py}

\myspace
\dirtree{%
.1 preprocessing.
.2 batch\_processing.
.2 chars\_mapping.
.2 data\_structs.
.2 files.
.2 to\_tensor.
.2 constants.py.
.2 exceptions.py.
.2 preprocessing.py <----.
}
\myspace

\begin{itemize}
	
\item  {Preprocessing [ klasa ] }
\begin{import}
from library.preprocessing.preprocessing import Preprocessing
\end{import}

Ta klasa, a raczej jej użycie zostało opisane w punkcie wyżej. Wszystkie elementy biblioteki opisane
poniżej są wykorzystywane przez tą klasę. Jest to klasa modułu preprocessingu przeznaczona
do domyślnego użycia.

\item  {CharactersMapper [ klasa ] }
\begin{import}
from library.preprocessing.preprocessing import CharactersMapper
\end{import}
Ta klasa, jest odpowiedzialna za redukcję alfabetu tekstów wejściowych. 

Przyjmuje takie argumenty jak: 

\begin{itemize}
	\item language
	\item data\_path
	\item mapped\_save\_path
	\item log\_lvl
\end{itemize}

Wszystkie te argumenty funkcojonują tak samo jak w klasie Preprocessing.


\item  {ToTensor [ klasa ] }
\begin{import}
from library.preprocessing.preprocessing import ToTensor
\end{import}
Ta klasa jest odpowiedzialna za zamianę tekstów na tensory.

Przyjmuje takie argumenty jak: 

\begin{itemize}
	\item language
	\item tensors\_path
	\item mapped\_source\_path
	\item batch\_size
	\item log\_lvl
	\item reduced\_authors
\end{itemize}

Wszystkie te argumenty z jednym wyjątkiem funkcjonują tak samo jak w klasie Preprocessing. 
Tym wyjątkiem jest argument \texttt{reduced\_authors}, które pozwala na bezpośrednie przekazanie obiektu z danymi 
zredukowanych tekstów do obiektu ToTensor. W ten sposób oszczędzany jest czas ładowania tekstów z 
dysku. Jeśli jest on podany, to argument \texttt{mapped\_source\_path} jest ignorowany.

\end{itemize}

\subsubsection{exceptions.py}

\myspace
\dirtree{%
.1 preprocessing.
.2 batch\_processing.
.2 chars\_mapping.
.2 data\_structs.
.2 files.
.2 to\_tensor.
.2 constants.py.
.2 exceptions.py <----.
.2 preprocessing.py.
}
\myspace
\begin{import}
from library.preprocessing import exceptions
\end{import}

W tym pliku znajdują się deklaracje wszystkich wyjątków używanych w module preprocessing

Są to:
\begin{itemize}
	\item WrongDataStructureException
	\item NotMappedDataException
	\item NoDataSourceSpecified
	\item NoLanguageSpecified
\end{itemize}

\newpage
\subsubsection{to\_tensor}

\myspace
\dirtree{%
.1 preprocessing.
.2 batch\_processing.
.2 chars\_mapping.
.2 data\_structs.
.2 files.
.2 to\_tensor <----.
.3 alphabets.
.4 en\_alphabet.py.
.4 es\_alphabet.py.
.4 gr\_alphabet.py.
.4 nl\_alphabet.py.
.3 convert.py.
.2 constants.py.
.2 exceptions.py.
.2 preprocessing.py.
}
\myspace

\begin{itemize}

\item{text\_to\_tensor [ metoda ] }

\begin{import}
from library.preprocessing.to_tensor.convert import text_to_tensor
\end{import}
	
Ta metoda jest odpowiedzialna za zamianę tekstów na tensory.

Przyjmuje takie argumenty jak: 

\begin{itemize}
	\item alphabet: List
	\item text: str
\end{itemize}

Szczegóły w dokumentacji w kodzie.

\item{char\_to\_tensor [ metoda ] }
\begin{import}
from library.preprocessing.to_tensor.convert import char_to_tensor
\end{import}
Ta metoda jest odpowiedzialna za zamianę znaków na wektory.

Przyjmuje takie argumenty jak: 

\begin{itemize}
	\item alphabet: List
	\item char: str
\end{itemize}

Szczegóły w dokumentacji w kodzie.


\item{wbudowane alfabety [ listy ] }

\begin{itemize} 

\item angielski
\begin{import}
from library.preprocessing.to_tensor.alphabets.en_alphabet import alphabet
\end{import}


\item hiszpański
\begin{import}
from library.preprocessing.to_tensor.alphabets.es_alphabet import alphabet
\end{import}


\item grecki
\begin{import}
from library.preprocessing.to_tensor.alphabets.gr_alphabet import alphabet
\end{import}


\item holenderski
\begin{import}
from library.preprocessing.to_tensor.alphabets.nl_alphabet import alphabet
\end{import}
	
\end{itemize}

\end{itemize}


\subsubsection{files}

\myspace
\dirtree{%
.1 preprocessing.
.2 batch\_processing.
.2 chars\_mapping.
.2 data\_structs.
.2 files <----.
.3 files\_operations.py.
.3 json\_loader.py.
.3 name\_convencion.py.
.2 to\_tensor.
.2 constants.py.
.2 exceptions.py.
.2 preprocessing.py.
}
\myspace

\begin{itemize}

\item{TextFileLoader [ klasa ] }

\begin{python}
from library.preprocessing.files.files_operations import TextFileLoader

tfl = TextFileLoader(path="path/to/file")
tfl.lines  # text as lines
tfl.text   # whole text
\end{python}

Po wywołaniu konstruktora ładuje do pamięci zawartość pliku spod podanej ścieżki. Mamy następnie 
dostęp do jej dwóch wersji - w postaci stringa lub listy (każdy element jest pojedynczą 
linią będącą stringiem)

\item{save\_to\_file [ metoda ] }
\begin{import}
from library.preprocessing.files.files_operations import save_to_file
\end{import}

Przyjmuje następujące argumenty:
\begin{itemize}
	\item path: str
	\item filename: str
	\item content: str
\end{itemize}

\item{check\_if\_directory [ metoda ] }
\begin{import}
from library.preprocessing.files.files_operations import check_if_directory
\end{import}
Zwraca wartość \texttt{True} jeśli element spod podanej ścieżki jest katalogiem, \texttt{False} w przeciwnym wypadku.
Przyjmuje następujące argumenty:
\begin{itemize}
	\item path: str
\end{itemize}

\item{check\_if\_file [ metoda ] }
\begin{import}
from library.preprocessing.files.files_operations import check_if_file
\end{import}
Zwraca wartość \texttt{True} jeśli element spod podanej ścieżki jest plikiem, \texttt{False} w przeciwnym wypadku.
Przyjmuje następujące argumenty:
\begin{itemize}
	\item path: str
\end{itemize}

\item{check\_if\_directory\_exists [ metoda ] }
\begin{import}
from library.preprocessing.files.files_operations import check_if_directory_exists
\end{import}
Jeśli katalog spod podanej ścieżki nie istnieje zgłasza wyjątek \texttt{FileNotFound}.
Przyjmuje następujące argumenty:
\begin{itemize}
	\item path: str
\end{itemize}

\item{create\_file [ metoda ] }
\begin{import}
from library.preprocessing.files.files_operations import create_file
\end{import}
Tworzy plik pod podaną ścieżką. 

Przyjmuje następujące argumenty:
\begin{itemize}
	\item alphabet: List
	\item char: str
\end{itemize}

\item{remove\_directory [ metoda ] }
\begin{import}
from library.preprocessing.files.files_operations import remove_directory
\end{import}

Usuwa katalog spod podanej ścieżki.
Przyjmuje następujące argumenty:
\begin{itemize}
	\item path: str
\end{itemize}

\end{itemize}

\newpage

\subsubsection{chars\_mapping}

\myspace
\dirtree{%
.1 preprocessing.
.2 batch\_processing.
.2 chars\_mapping <----.
.3 mappers.
.4 en\_mapper.py.
.4 es\_mapper.py.
.4 gr\_mapper.py.
.4 nl\_mapper.py.
.3 map.py.
.2 data\_structs.
.2 files.
.2 to\_tensor.
.2 constants.py.
.2 exceptions.py.
.2 preprocessing.py.
}
\myspace

\begin{itemize}

\item{map\_characters [ metoda ] }
\begin{import}
from library.preprocessing.chars_mapping.map import map_characters
\end{import}

Redukuje znaki w podanym stringu na podstawie słownika. 
Przyjmuje następujące argumenty:
\begin{itemize}
	\item mapper: dict
	\item text: str
\end{itemize}


\item{wbudowane mappery [ słowniki ] }

\begin{itemize} 

\item angielski
\begin{import}
from library.preprocessing.chars_mapping.mappers.en_mapper import charmap
\end{import}


\item hiszpański
\begin{import}
from library.preprocessing.chars_mapping.mappers.es_mapper import charmap
\end{import}


\item grecki
\begin{import}
from library.preprocessing.chars_mapping.mappers.gr_mapper import charmap
\end{import}


\item holenderski
\begin{import}
from library.preprocessing.chars_mapping.mappers.nl_mapper import charmap
\end{import}

\end{itemize}

\end{itemize}

\newpage
\subsubsection{batch\_processing}

\myspace
\dirtree{%
.1 preprocessing.
.2 batch\_processing <----.
.3 batch.py.
.2 chars\_mapping.
.2 data\_structs.
.2 files.
.2 to\_tensor.
.2 constants.py.
.2 exceptions.py.
.2 preprocessing.py.
}
\myspace

\begin{itemize}

\item {BatchProcessor [ klasa ] }
\begin{import}
from library.preprocessing.batch_processing.batch import BatchProcessor
\end{import}

\TODO {uzupełnić}

\end{itemize}