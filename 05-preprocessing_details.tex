\newpage
\subsection{Budowa}
W tej części opisana zostanie budowa modułu Preprocessingu - z jakich fragmentów się składa
oraz za co one odpowiadają. Opisane zostaną tylko ważniejsze fragmenty kodu z racji na jego 
obszerność.

\subsubsection{Preprocessing [ klasa ] }
\begin{import}
from library.preprocessing.preprocessing import Preprocessing
\end{import}

Ta klasa, a raczej jej użycie zostało opisane w punkcie wyżej. Wszystkie elementy biblioteki opisane
poniżej są wykorzystywane przez tą klasę. Jest to klasa modułu preprocessingu przeznaczona
do domyślnego użycia.

\subsubsection{CharactersMapper [ klasa ] }
\begin{import}
from library.preprocessing.preprocessing import CharactersMapper
\end{import}
Ta klasa, jest odpowiedzialna za redukcję alfabetu tekstów wejściowych. 

Przyjmuje takie argumenty jak: 

\begin{itemize}
	\item language
	\item data\_path
	\item mapped\_save\_path
	\item log\_lvl
\end{itemize}

Wszystkie te argumenty funkcojonują tak samo jak w klasie Preprocessing.


\subsubsection{ToTensor [ klasa ] }
\begin{import}
from library.preprocessing.preprocessing import ToTensor
\end{import}
Ta klasa, jest odpowiedzialna za zamiane tekstów na tensory.

Przyjmuje takie argumenty jak: 

\begin{itemize}
	\item language
	\item tensors\_path
	\item mapped\_source\_path
	\item batch\_size
	\item log\_lvl
	\item reduced\_authors
\end{itemize}

Wszystkie te argumenty z jednym wyjątkiem funkcjonują tak samo jak w klasie Preprocessing. 
Tym wyjątkiem jest reduced\_authors, które pozwala na bezpośrednie przekazanie obiektu z danymi 
zredukowanych tekstów do obiektu ToTensor, w ten sposób oszczędzany jest czas ładowania tekstów z 
dysku. Jeśli jest on podany argument mapped\_source\_path jest ignorowany.



\subsubsection{text\_to\_tensor [ metoda ] }
\begin{import}
from library.preprocessing.to_tensor.convert import text_to_tensor
\end{import}
Ta metoda, jest odpowiedzialna za zamiane tekstów na tensory.

Przyjmuje takie argumenty jak: 

\begin{itemize}
	\item alphabet: List
	\item text: str
\end{itemize}

Szczegóły w dokumentacji w kodzie.

\subsubsection{char\_to\_tensor [ metoda ] }
\begin{import}
from library.preprocessing.to_tensor.convert import char_to_tensor
\end{import}
Ta metoda, jest odpowiedzialna za zamiane znaków na wektory.

Przyjmuje takie argumenty jak: 

\begin{itemize}
	\item alphabet: List
	\item char: str
\end{itemize}

Szczegóły w dokumentacji w kodzie.


\subsubsection{wbudowane alfabety [ listy ] }

\begin{itemize} 

\item angielski
\begin{import}
from library.preprocessing.to_tensor.alphabets.en_alphabet import alphabet
\end{import}


\item hiszpański
\begin{import}
from library.preprocessing.to_tensor.alphabets.es_alphabet import alphabet
\end{import}


\item grecki
\begin{import}
from library.preprocessing.to_tensor.alphabets.gr_alphabet import alphabet
\end{import}


\item holenderski
\begin{import}
from library.preprocessing.to_tensor.alphabets.nl_alphabet import alphabet
\end{import}
\end{itemize}


\subsubsection{TextFileLoader [ klasa ] }
Po wywołaniu konstruktora ładuje do pamięci zawartośc pliku spod podanej ścieżki. Mamy następnie 
dostęp do jej dwóch wersji - w postaci stringa lub listy (każdy element jest pojedynczą 
linią będącą stringiem)


\begin{python}
from library.preprocessing.files.files_operations import TextFileLoader

tfl = TextFileLoader(path="path/to/file")
tfl.lines  # text as lines
tfl.text   # whole text
\end{python}

\subsubsection{save\_to\_file [ metoda ] }
\begin{import}
from library.preprocessing.files.files_operations import save_to_file
\end{import}

Przyjmuje następujące argumenty:
\begin{itemize}
	\item path: str
	\item filename: str
	\item content: str
\end{itemize}

\subsubsection{check\_if\_directory [ metoda ] }
\begin{import}
from library.preprocessing.files.files_operations import check_if_directory
\end{import}
Zwraca True jeśli element spod podanej ścieżki jest katalogiem, False w przeciwnym wypadku.
Przyjmuje następujące argumenty:
\begin{itemize}
	\item path: str
\end{itemize}

\subsubsection{check\_if\_file [ metoda ] }
\begin{import}
from library.preprocessing.files.files_operations import check_if_file
\end{import}
Zwraca True jeśli element spod podanej ścieżki jest plikiem, False w przeciwnym wypadku.
Przyjmuje następujące argumenty:
\begin{itemize}
	\item path: str
\end{itemize}

\subsubsection{check\_if\_directory\_exists [ metoda ] }
\begin{import}
from library.preprocessing.files.files_operations import check_if_directory_exists
\end{import}
Jeśli katalog spod podanej ścieżki nie istnieje rzuca wyjątek FileNotFound.
Przyjmuje następujące argumenty:
\begin{itemize}
	\item path: str
\end{itemize}

\subsubsection{create\_file [ metoda ] }
\begin{import}
from library.preprocessing.files.files_operations import create_file
\end{import}
Tworzy plik pod podaną ścieżką. 

Przyjmuje następujące argumenty:
\begin{itemize}
	\item alphabet: List
	\item char: str
\end{itemize}

\subsubsection{remove\_directory [ metoda ] }
\begin{import}
from library.preprocessing.files.files_operations import remove_directory
\end{import}

Usuwa katalog spod podanej ścieżki.
Przyjmuje następujące argumenty:
\begin{itemize}
	\item path: str
\end{itemize}
