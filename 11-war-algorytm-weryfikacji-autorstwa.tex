\subsection{Algorytm identyfikacji autorstwa tekstu}

Do wielogłowicowej sieci wprowadzamy w jednym momencie wielu autorów, pochodzących ze zbioru 
testowego w postaci paczek danych (batche). Oprócz danych moduł batchowania dostarcza również 
oczekiwanie wyjście, któreposłuży do obliczania wartości funkcji kosztu oraz ewaluacji skuteczności. 
Sieć składa się z określonej przez parametr liczby warstw ukrytych oraz warstwy wyjściowej, 
liniowej, w pełni połączonej, która mapuje wektor stanu na wyjście. Następnie na podstawie 
Krosentropii wyliczany jest loss, który odzwierciedla jak bardzo otrzymane wyjście różni się od oczekiwanych 
wyników. Na podstawie wartości loss dla każdego autora dokonoywana jest propagacja wsteczna. 
Z wykorzystaniem algorytmu optymalizacji uaktualniane są wagi. 
\newline
\newline
Do badania skuteczności sieci wykorzystywany jest zbiór walidacyjny. Do modelu wprowadzane są dane, 
następnie otrzymane wyniki są sprowadzane do wartości z przedziału [0,1], 
gdzie wynik bliski 0 oznacza najmniej pasującego autora, a bliski 1 najbardziej dopasowanego.
Dzięki temu możemy stwierdzić, którego autora przeiwduje sieć i zestawić to z prawdziwym wynikiem.
Skuteczność opisana jest jako stosunek poprawnych odpowiedzi do wszystkich. Niska skuteczność może 
świadczyć o źle dobranych parametrach sieci bądź niepoprawnie przeprowadzonym procesie uczenia.