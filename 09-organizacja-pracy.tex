\newpage
\section{Organizacja pracy}

\subsection{Metodyka}

Prace nad projektem były prowadzone zgodnie z iteracyjnym modelem wytwarzania oprogramowania. 
Większość iteracji opierała się na następującym schemacie: 
\begin{enumerate}
  \item wybranie funkcjonalności do zaimplementowania,
  \item zdobycie informacji na temat tej funkcjonalności,
  \item implementacja funkcjonalności,
  \item testowanie funkcjonalności,
  \item stworzenie dokumentacji na temat tej funkcjonalności.
\end{enumerate}
Każda iteracja, ze względu na odmienny charakter trwała inną ilość czasu. Najdłuższa okazała się
pierwsza i ostatnia iteracja. W skład pierwszej wchodziło zapoznanie się z tematyką sieci typu RNN oraz
jej implemetacja, w skład ostatniej uczenie sieci oraz szukanie najlepszych parametrów sieci do problemu 
identyfikacji autorstwa tekstu.

\subsection{Role i podział prac}
Bazując na zadaniach wykonywanych podczas pracy nad biblioteką, można wyróżnić następujące role:
\begin{itemize}
  \item Programista - implementacja kodu
  \item Analityk -  bezpośredni kontakt z klientem i określanie wymagań odnośnie tworzonej biblioteki
  \item Tester - testowanie kodu napisanego przez programiste
  \item Osoba odpowiedzialna za tworzenie dokumentacji
\end{itemize}
Projekt wykonywany był przez dwie osoby:
\begin{itemize}
  \item Jakub Ziarko - implementacja modułu preprocessingu, implementacja modułu sieci RNN, 
  trenowanie sieci, pisanie dokumentacji, stworzenie skrytpów ułatwiających prace na Prometheusu.
  \item Jan Liberacki - implementacja modułu odpowiedzialnego za podział danych na batche, 
  implementacja modułu sieci RNN, trenowanie sieci, pisanie dokumentacji, testowanie kodu.
\end{itemize}
Opiekunowie pracy:

\begin{itemize}
  \item dr inż. Kuta Marcin - główny opiekun pracy, pełniący rolę klienta końcowego i osoby nadzorującej
  postęp prac nad realizacją i implementacją projektu. Z tą osobą zespół konsultował wszelkie wątpliwości
  natury technicznej. Dzięki jego uwagom doprowadzono do wielu usprawnień w zakresie działania sieci.
  \item dr inż. Rafał Dreżewski - prowadzący Pracownię Projektową, nadzorował postęp prac, 
  szczególnie ten w dokumentacji. Dzięki jego uwagom udało się dokonać wielu usprawnień w zakresie podziału
  dokumentacji oraz jej formy.
\end{itemize}

\subsection{Komunikacja z opiekunem pracy}
W trakcie pracy nad biblioteką odbywały się regularne spotkania Pracowni Projektowej, w ramach której 
przedstawiane były kolejne postępy pracy nad biblioteką. W ramach Pracowni Projektowej byliśmy również
zobligowani do dostarczania kolejnych części dokumentacji na określone terminy. Spotkania z głównym
opiekunem pracy wyznaczało tempo pisania kolejnych modułów. Często w ramach przejścia do kolejnej iteracji
pojawiały się nowe problemy, które wymagały pomocy głównego opiekuna. Większość pytań kierowanych do 
głównego opiekuna była realizowana drogą mailową. 

\subsection{Narzędzia}
Sprawna realizacja projektu wymagała użycia odpowiednich narzędzi. Wśród wielu wykorzystywanych narzędzi,
chcielibyśmy wyróżnić te najważniejsze:

\begin{itemize}
  \item PyCharm - zintegrowane środowisko programistyczne dla języka Python od firmy JetBrains.
  Wybrane przez nas ze względu na nasze wcześniejsze pozytywne doświadczenia z tym środowiskiem.
  \item Git - rozporoszony system kontroli wersji. Zdecydowaliśmy się na niego ponieważ ze względu na 
  dwuosobowy zespół zależało nam na możliwości współbieżnej pracy nad kodem i dokumentacją.
  \item GitHub - służył jako repozytorium do przechowywania dokumentacji oraz kodu.
  \item Python - język programowania. Wybrany ze względu na wysokopoziomowość oraz dostępność biblioteki Pytorch.
  \item PyTorch - otwartoźródłowa biblioteka programistyczna języka Python do uczenia maszynowego.
  \item Facebook - główne narzędzie komunikacji pomiędzy członkami zespołu.
  \item Latex - oprogramowanie do zautomatyzowanego składu tekstu, a także związany z nim język 
  znaczników, służący do formatowania dokumentów tekstowych i tekstowo-graficznych. 
  \item Eclipse - zintegrowane środowisko programistyczne do tworzenia programów, wybrane w celu tworzenia 
  dokumentacji.
  \item TeXlipse - rozbudowana wtyczka do środowiska Eclipse ułatwiająca pisanie w Latex.  
  \item Matplotlib - biblioteka Pythonowa do rysowania wykresów, wykorzystana przez nas do analizy danych 
  z procesu uczenia sieci. 
\end{itemize}

\subsection{Zastosowane techniki i praktyki}
Realizacji biblioteki towarzyszyło zastosowanie kilku technik i praktyk mających na celu efektywne 
tworzenie kodu o dobrej jakości:
\begin{itemize}
  \item refactoring kodu - nie wszystkie moduły biblioteki od początku napisane były dobrze. Niekiedy z racji 
  na kolejne optymalizacje bądź korekcje w sposobie uczenia sieci należało szybko coś zmienić w kodzie,
  co z czasem owocowało brakiem porządku i kodem niskiej jakości. Okresowe refactoringi pozwoliły temu zapobiec.
  \item pull request oraz code review - ze względu na złożoność modułów i potrzebe utrzymywania wysokiej
  jakości kodu zmiany wprowadzane w repozytorium były zazwyczaj poprzez tworzenie pull requestów i code review
  drugiego członka zespołu. W ten sposób obaj na bieżąco zapoznawaliśmy się ze zmianami oraz zwiększaliśmy
  jakość kodu, poprzez efektywniejsze wychwytywanie błędów oraz złych wzorców programowania.
\end{itemize}

\subsection{Postępy}
\TODO{uzupełnić} 

\subsection{Wykonane prace}
W trakcie realizacji pracy inzżynierskiej wykonaliśmy następujące prace:
\begin{itemize}
  \item stworzenie modułu preprocessingu - moduł który w wygodny sposób pozwala nam na przygotowanie tekstów 
  do uczenia maszynowego. 
  \item stworzenie modułu batchowania - moduł, który bazując na danych przygotowanych przez moduł 
  preprocessingu dzieli je na batche i dostarcza sieci. Jest on sparametryzowany dzięki czemu mogliśmy
  w łatwy sposób wpływać na parametry batchy i efektywnie szukać tych najlepszych. 
  \item moduł sieci - moduł, a właściwie skrypt który pozwala na zdefiniowanie rekurencyjnej sieci neuronowej
  do problemu idetyfikacji autorstwa tekstu. Współpracuje bezpośrednio z modułem batchowania.
  \item moduł do przetwarzania wyjścia z sieci - odpowiedzialny za zbieranie informacji podczas procesu uczenia,
  następnie wybieranie częsci z nich i umieszczanie ich w uporządkowanej formie w pliku csv oraz 
  jako standardowe wyjście. Był on równiez odpowiedzialny za zapisywanie kolejnych wytrenowanych modeli. 
  \item skrypty do pracy na Prometheushu:
  \begin{itemize}
    \item skrypt do wysyłania przetworzonych tekstów na Prometheusha 
    \item skrypt do kopiowania danych po procesie uczenia maszynowego z Prometheusha na komputer osobisty
    \item skrypt do rysowania wykresów na podstawie danych zerbanych w procesie uczenia sieci
    \item skrypt do definiownia parametrów sieci, których szkolenie było następnie w sposób z
    automatyzowany przeprowadzane na Prometheusu. Dzięki niemu nie musieliśmy pilnować procesu uczenia i 
   	każde z nich przeprowadzać osobno. Definiowaliśmy parametry wielu sieci, które chcieliśmy przetestować,
   	następnie skrypt generował kod który kolejkował na Prometheuszu uczenie wszystkich wyszczególnionych przez nas sieci 
   	jako kolejne zadania.
  \end{itemize}
  \item wykonanie niniejszej dokumentacji
\end{itemize}


\subsection{Problemy napotkane w trakcie realizacji projektu}
\TODO{uzupełnić} 
